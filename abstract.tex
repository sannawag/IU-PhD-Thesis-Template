%%%%%%%%%%%%%%%%%%%%%%%%%%%%%%%%%%%%%%%%%%%%%%%%%%%%%%%%%
% Do not edit these lines unless you wish to customize
% the template
%%%%%%%%%%%%%%%%%%%%%%%%%%%%%%%%%%%%%%%%%%%%%%%%%%%%%%%%%
\newgeometry{left=1in}

\begin{center}

\yourName\\
\MakeUppercase{\thesisTitle}

\end{center}

\vspace{1.5\baselineskip}

%Insert your abstract here



In this thesis, I introduce a data-driven algorithm for pitch correction of monophonic singing performances. The algorithm uses reference pitches in the accompaniment track to predict shifts in the vocals track. It involves a deep neural network model that learns musical intonation patterns from 4,702 amateur karaoke performances selected for accuracy. The model is exposed both to incorrect intonation, for which it learns a correction, and intentional pitch variation, which it learns to preserve. The proposed algorithm represents pitch as a continuous value, differing from existing commercial systems, where vocal track notes are usually mapped to one of the twelve equal-tempered scale degrees. In this thesis, I argue---based on evidence from musical studies and from theoretical writings on musical intonation---that few musical traditions fit the equal-tempered scale. Deviations from the scale are often a crucial musical component, and the model can be trained on music from a given culture to preserve its deviations. I argue further in favor of a data-driven approach to digital music processing over a model-based approach, prioritizing expressivity over control and interpretability. The proposed deep neural network with gated recurrent units on top of convolutional layers shows promising performance on the real-world score-free singing pitch correction task. 



\ifdefined\committeeMemberFourTypedName

\null\hfill \myRule\\
\null\hfill \committeeChairpersonTypedName, \committeeChairpersonPostNominalInitials\\
\null\hfill \myRule\\
\null\hfill \committeeMemberTwoTypedName, \committeeMemberTwoPostNominalInitials\\
\null\hfill \myRule\\
\null\hfill \committeeMemberThreeTypedName, \committeeMemberThreePostNominalInitials\\
\null\hfill \myRule\\
\null\hfill \committeeMemberFourTypedName, \committeeMemberFourPostNominalInitials\\

\ifdefined\committeeMemberFiveTypedName
\null\hfill \myRule\\
\null\hfill \committeeMemberFiveTypedName, \committeeMemberFivePostNominalInitials\\
\fi

\fi
\restoregeometry

