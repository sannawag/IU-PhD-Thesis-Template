\chapter{Conclusion and future work}
\label{chap:conclusion}

In this thesis, I introduce a novel data-driven algorithm for estimating and applying pitch corrections in a monophonic vocal track, while using the backing track (also called accompaniment) as a reference. The deep neural network used to predict pitch corrections is exposed to both incorrect intonation, for which it learns a correction, and intentional pitch variation, which it learns to preserve. It treats pitch as a continuous value rather than a discrete set of notes. It does not rely on a musical score, thus allowing for improvisation and harmonization in the singing performance. Results on a convolutional neural network with a gated recurrent unit layer indicate that spectral information in the backing and vocal tracks is useful for determining the amount of pitch correction required at the note level. 

The fact that musical pitch isn't represented as a discrete set of notes also means that the algorithm applies to many different musical traditions, regardless of the scales and pitch variation patterns used in these traditions. Though the network was trained in this thesis work on Western popular music---as this was the data that was available---it can be trained on music from other genres and cultures.

The amateur ``Intonation'' dataset used as ground truth in this thesis has a very different pitch distribution from that of a professional dataset. For this reason, the results described in this thesis are prototypical in nature. One of the challenges for future work is to utilize professional-level data to make the predictions as accurate as possible. 

The current model is built on some strong assumptions. First, that the backing track has clearly identifiable pitches---a chord progression---which serve as a reference for the vocals. Second, that a note's pitch can be corrected by shifting the full note by a constant. Moving beyond these assumptions can both make the model usable in more contexts and make it more accurate in its current context. 

The current algorithm first predicts the amount by which singing should be shifted in pitch, then applies the shift in post processing. In future work, the model can be extended to directly predict the pitch-shifted signal in an end-to-end model. 

A machine-learning-based algorithm can always make mistakes. Developing a practical tool based on this algorithm will require collaboration with human-computer interaction designers, music theorists, and musicologists.

\section{Broader impacts}
The work in this thesis contributes to the recent field of music information retrieval, which lies in the intersection of music and technology. The automatic pitch correction algorithm introduced in this thesis is an illustration of how recent technological advances can be used in the service of music. It also provides an example of how music technology can be developed in the context of millenia of music theory, and how technology can provide new means for artistic expression and development.

Digital apps may be the primary way that many people interact actively with music, or the first approach people try when getting started with music. The quality of the experience and its potential to lead to musical growth may determine whether a user keeps making music or gives it up.
