% Define your dedication statement here

\newcommand{\yourDedication}{To my family.}

%%%%%%%%%%%%%%%%%%%%%%%%%%%%%%%%%%%%%%%%%%%%%%%%%%%%%%%%%
% Do not edit these lines unless you wish to customize
% the template
%%%%%%%%%%%%%%%%%%%%%%%%%%%%%%%%%%%%%%%%%%%%%%%%%%%%%%%%%

\begin{center}

\vspace*{\fill}
\yourDedication\\
%Multiple design decisions lie behind any sound that reaches our ears. Given the current design of processors, contDesign starts with the way a continuous, acoustic signal is represented as a discrete sequence of  is built on multiple design decision, ranging from the 
%We interact with digital music on a regular basis, through music recordings and streaming, electric instruments, and tools such as microphones and recording devices.  


%Through college, my music making was first and foremost in the acoustic realm. I was trained in the Western Classical tradition, first on piano, then on bassoon. Even then, I interacted with digital music on a daily basis, through the radio, music streaming apps, electric keyboards, and tools such as microphones, tuners, and recording devices. In college, I grew curious about the representations of music as data in the digital realm. I became aware of how behind all of them were design decisions by audio and software engineers. I wished to understand how these representations could affect our experience of music, for the better, for the worse, or some of both. Conveniently, a graduate degree existed for this, named music informatics. 

%The question arises of what place apps hold in the world of music making. Apps can easily come across as toys, too musically limiting to enable the user's lifelong pursuit of excellence that is music learning. At the 2017 Audio Developer Conference keynote,\footnote{``Julian Storer - Keynote: Does your code actually matter? (ADC’17).'' YouTube Video, 0:00. Posted November 17, 2017. https://youtu.be/Yd0Ef6uzJb0} Julian Storer described how even engineers working on digital audio processing tools for professional musicians sometimes wondered whether they were building gadgets.\footnote{Thank you to Google and my intern host, Glenn Kasten, for funding my trip to the Audio Developer Conference.}

%During my internship with Smule, Inc.---a singing app---in 2018, I learned how some users will take a ten-minute break in their car in the middle of a workday to sing. Many have written to the company to thank it for bringing them joy and relief from stress. For some people, these ten-minute breaks are all the music that they have the time to make on a regular basis. This is one example of how apps have the potential to play a significant role in one's musical experience, and that with high-quality technology these experiences can be musically rich. Having personally started singing during graduate school, I found that Smule provided a good framework for learning songs, and was inspired by the experience. For some, an app can serve as an entryway into music. Especially in the case of people who do not grow up with much exposure to music, or do not have acoustic musical instruments easily available, digital audio and apps might be the main way to interact with music. The quality of the sound and musical richness in these apps might even influence whether a person develops a deeper interest in music making or gives it up. 

\vspace*{\fill}

\end{center}